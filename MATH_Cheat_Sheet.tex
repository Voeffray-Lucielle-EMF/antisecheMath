\documentclass{article}
\title{Antisèche de Math}
\author{Anya Voeffray \thanks{thanks to no fucking one, I hate Maths}}
\date{Octobre 2024}

\DeclareUnicodeCharacter{2212}{-}

\begin{document}
	
	
\begin{titlepage}
	
	\maketitle
	
	\begin{equation}
		Mes Capacites En Math =  \frac{Motivation \cdot LogiqueMathematique}{Annee Depuis Le Dernier Cours De Maths}
	\end{equation}
	
\end{titlepage}

\section{Identités remarquables}

\begin{equation}
	(a+b)^2 = a^2 + 2ab + b^2
\end{equation}

Un exemple:
\begin{equation}
	(2 + 3)^2 = 2^2 + 2\ast2\ast3 + 3^2
\end{equation}
\begin{equation}
	(2 + 3)^2 = 4 + 12 + 9 = 25
\end{equation}

\begin{equation}
	(a−b)^2 = a^2 − 2ab + b^2
\end{equation}

\begin{equation}
	(a+b)\ast(a−b) = a^2 − b^2
\end{equation}

\section{Isolation des inconnues}

TODO
	
\end{document}